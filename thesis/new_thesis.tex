\makeatletter
\def\sx@stop{\relax}\def\sx@open{(}\def\sx@close{)}
\def\sx@scan#1{%
  \def\sx@arg{#1}%        save the current token
  \ifx\sx@arg\sx@stop   % is it the end of the list?
    \let\sx@next\relax
  \else
    \ifx\sx@arg\sx@close\else\sx@space\fi % add space unless token is ")"
    {\sx@arg}%                            % write the token
    \ifx\sx@arg\sx@open                   
      \let\sx@space\relax                 % kill next space if token is "("
    \else
      \let\sx@space\;
    \fi
    \let\sx@next\sx@scan                  % scan next token
  \fi\sx@next
}
\newcommand{\sexp}[1]{\let\sx@space\relax\sx@scan#1\relax}
\makeatother

\title{S-expressions as a means for audio synthesis}
\author{
Chris Donahue \\
Advisors: Peter Stone, Russell Pinkston \\
Department of Computer Science \\
University of Texas at Austin\vspace{-2ex}
}
\date{}

\documentclass[12pt]{article}

\usepackage{natbib}
\usepackage{verbatim}
\usepackage{amsmath}
\usepackage{hyperref}

\newcommand{\audiolanguagenamelower}{\emph{synthax}}
\newcommand{\audiolanguagenameupper}{\emph{Synthax}}

\begin{document}
\maketitle

\begin{abstract}
This paper presents an alternative method of digital audio synthesis using Lisp-like symbolic expressions (S-expressions). The method centers around a novel audio programming language called \audiolanguagenamelower{} where valid programs directly represent algorithms for audio synthesis. The motivation for \audiolanguagenamelower's use of S-expressions to generate audio is that they are recursively defined, flexible and convenient from a computational perspective. Additionally, the structures can be created and manipulated through genetic programming allowing users to search for synthesis algorithms that mimic existing instruments or to find entirely new timbres.
\end{abstract}



\section{Introduction}\label{TOPINTRO}
This paper will begin by elaborating on the \audiolanguagenamelower{} programming language. Following this two concrete applications of the language will be presented. The first application is a system for automatically generating audio synthesis algorithms with the goal of mimicking the timbre of a provided sound file. The second application is a VST plugin that allows users to interactively evolve and control \audiolanguagenamelower{} programs using MIDI. Finally, there will be a discussion of future applications and ideas for expanding the language.

\section{The Language}

\subsection{Symbolic Expression of Audio Synthesis Algorithms}\label{SEXPRESSION}

\begin{equation}\label{SINRENDERINFIX}
sin (2 \times \pi \times f \times t)
\end{equation}

\begin{figure}\label{SINRENDERNETWORKFIG}
	\caption{Traditional audio synthesis algorithm diagram of sine wave rendering equation}
	\centering
\end{figure}

\begin{equation}\label{SINRENDERSEXP}
\sexp{({sin} (\times (\times (\times 2 \pi) f) t))}
\end{equation}

\begin{figure}\label{SINRENDERSEXPFIG}
	\caption{S-expression diagram of sine wave rendering equation}
	\centering
\end{figure}

Audio synthesis algorithms (ASA's) are numerical algorithms designed to fill an audio buffer with a changing signal. Perhaps the quintessential ASA is the algorithm for rendering a pure sine wave tone. Mathematically, this equation can be expressed as \ref{SINRENDERINFIX} where $f$ is the desired frequency of the waveform and $t$ is the time in seconds. The value of $t$ at the $n$th sample would be $n/sample rate$. To render this structure to an audio sample buffer, one would calculate the output of the equation at the desired frequency $f$ and at values of $t$ corresponding to the start times of the samples.

The sine wave rendering equation can be expressed in several different ways. It is displayed as an infix equation in \ref{SINRENDERINFIX} and as a synthesis diagram in \ref{SINRENDERNETWORKFIG}. It can be also be expressed equivalently as a traditional symbolic expression as shown in text form in \ref{SINRENDERSEXP} and image form in \ref{SINRENDERSEXPFIG}.

Expressions such as the sine wave rendering equation are so important to the domain of musical sounds that it will be convenient for our purposes to see them as single units. This allows for a larger amount of programs in \audiolanguagenamelower{} to produce human-recognizable audio than would be the case if it was instead necessary to build the eight-unit subtree shown in \ref{SINRENDERSEXPFIG} to render a sine wave.

\subsection{Syntax}\label{SYNTAX}
\begin{description}\label{CURRENTPRIMITIVES}
	\item[first] \hfill \\
		The first item
	\item[second] \hfill \\
		the second item
\end{description}

\begin{equation}\label{SYNTAXEXAMPLE}
\sexp{({function} {name} \{{c/d} {min} {val} {max}\}\ldots ({child} {function})\ldots)}
\end{equation}

\audiolanguagenameupper{} is a functional programming language with a symbolic expression-based syntax. Legal expressions begin with a function name followed by zero or more numeric parameters and zero or more child expressions as shown in \ref{SYNTAXEXAMPLE}. The current functions in \audiolanguagenamelower{} are listed in \ref{CURRENTPRIMITIVES}.

\subsection{Rendering Audio From \audiolanguagenameupper{} Code}\label{RENDER}
Audio is generated from \audiolanguagenamelower{} programs recursively using a post-order traversal. Any given expression must render all audio signals from its children before rendering itself. The audio signal trickles up from the leaves towards the root node where the final output is rendered. At the present time \audiolanguagenamelower{} programs are strictly trees; no expression can have more than one parent.

In order to maximize efficiency the object representation of an expression must know certain conditions about the environment before rendering. These necessary conditions are the sample rate, the render block size, and the maximum render length in seconds. These allow the expressions to allocate necessary memory in advance of rendering. Expressions are also required to keep track of their minimum and maximum output values as parent expressions may need to transform their childrens' ranges to a different range.

This rendering paradigm is highly parallelizable. It is possible to render any sub-expression in parallel with others by allocating one additional floating-point buffer the size of the render block. Rendering can be distributed amongst a pool of worker threads evaluating progressively shallower sub-expressions. Allocating the worker threads to sub-expressions with multiple child nodes may be the most efficient way to approach parallelization. At present time no parallel rendering implementation exists for \audiolanguagenamelower, though it is certainly theoretically possible.




\section{Application 1: Automatic Timbre Mimicking}

\subsection{Introduction}\label{TMINTRO}
Timbre is a collection of physical properties of a sound and is frequently explained in terms of its human utility rather than its underlying components. The American National Standards Institute defined timbre as ``that attribute of auditory sensation in terms of which a listener can judge that two sounds similarly presented and having the same loudness and pitch are dissimilar'' \citep{american1960american}. The ambiguity lies in determining what exactly ``that attribute'' is and what physical phenomena cause it. This psychoacoustic debate continues to this day; the advent of computer technology has introduced both a road block and an opportunity in our understand of timbre \citep{erickson1975sound}. Computers have the ability to render any sound representable by a discrete waveform and thus the umbrella term timbre can no longer be interpreted as a direct association between certain sounds and the concrete characteristics of the objects that produce them.

Early attempts to characterize the physical components of timbre mostly focused on the relative amplitudes and ratios of partial tones. These assumptions were based perhaps too heavily on Fourier's theorem that any periodic waveform can be decomposed into simple sin wave components \citep{helmholtz1857physiological, helmholtz1954sensations}. Like most aspects of music however, humans interpret not the absolute values of individual notes or frequencies but the way their relations change over time. Characterizations of timbre that place full emphasis on instantaneous spectral relationships are antiquated and have been augmented by multidimensional descriptions made up of several components with codependencies \citep{erickson1975sound}. One such description was proposed by J. F. Schouten in 1968 and involves five characteristics \citep{schouten1968perception, erickson1975sound}:
\begin{enumerate}
\item
The range between tonal and noise-like character
\item
The spectral envelope
\item
The time envelope in terms of rise, duration and decay
\item
The changes both of spectral envelope (formant-glide) and fundamental frequency (micro-intonation)
\item
The prefix, an onset of a sound quite dissimilar to the ensuing lasting vibration.
\end{enumerate}

For the purposes of this application, these five perceptual characteristics will be used as guidelines to develop a method for comparing the timbral characteristics of two sounds. A computable metric of timbral comparison that is similar to human judgement is necessary for a genetic algorithm designed to produce synthesis algorithms capable of believably imitating target timbres.



\subsection{Motivations}\label{TMMOTIVATION}
Chowning summarized the motivations for timbre matching research in his paper detailing the technique of frequency modulation audio synthesis.
\begin{quote}
Of interest in both acoustical re-search and electronic music is the synthesis of natural sound. For the researcher, it is the ultimate test of acoustical theory, while for the composer of electronic music it is an extraordinarily rich point of departure in the domain of timbre, or tone quality. \citep{chowning1973synthesis}
\end{quote}
Acoustic instruments are something of a rigor test for any new synthesis technique. They represent well-known ``problems'' that can be used to demonstrate the adaptability of any new synthesis algorithm and the field of sounds that it can create. In the past, there have been many attempts to solve these ``problems'' by hand. This paper will present a system that allows one to supply a target sound and automatically receive a synthesis algorithm that mimics its timbral characteristics.
	

\subsection{Historical Precendence}\label{TMHISTORY}
Audio synthesis algorithms are a well studied aspect of computer music, and for as long as they have existed people have been trying to use them to mimic recognizable acoustic timbres.

\subsubsection{Early Investigations (1950-1970)}
Max Mathews created the first widely used computer program for audio synthesis in 1957 called MUSIC. MUSIC and its descendants in the MUSIC-N family of programs operate on high level audio generators and functions. Mathews wrote that ``there are no theoretical limitations to the performance of the computer as a source of musical sounds, in contrast to the performance of ordinary instruments.'' \citep{mathews01111963} While this statement can be challenged due to a computer's lack of ability to produce continuous signals, it holds perceptually as any discrete wave form can be rendered by a computer. Jean-Claude Risset worked with Mathews in the late 1960's to attempt to produce with a computer sounds that mimicked acoustic instruments. While they were successful in creating brass-like and string-like sounds by analyzing the spectral content of recordings, their conclusion was that ``additional factors must be discovered to yield an impeccable match to a given [instrument]'' \citep{risset1969analysis}.

\subsubsection{FM Synthesis (1970-1990)}
In 1973 John Chowning introduced a method for utilizing frequency modulation to efficiently synthesize sounds with rich harmonic spectra that developed over time. He states in the introduction of the related paper that ``temporal evolution of the spectral components is of critical importance in the determination of timbre'' \citep{chowning1973synthesis}. The task of efficiently producing spectrally complex tones had presented a major hurdle to mimicking spectrally complex acoustic instruments as computer performance was limited. Chowning himself used FM synthesis to mimic brass, woodwind, percussive sounds, and later a singing voice \citep{chowning1989frequency} with impressive subjective results. Perhaps one of the ``additional factors'' elusive to Mathews and Risset was the temporal evolution of spectral components that FM synthesis provides.

Work in the 1980's on mimicking acoustic timbres was largely commercial after Stanford licensed Chowning's work on FM audio synthesis to Yamaha. Audio researchers at Yamaha continued to mimic tones of acoustic instruments by modifying the parameters and carrier/modulation structure of FM synthesis.

\subsubsection{Genetic Algorithms (1990-present)}
\paragraph{Traditional Genetic Algorithms}
Genetic algorithms (GA's) were only popularized in the 1980's despite previous decades of theoretical study. GA's emulate the natural process of evolution over a population of genotypes, assigning higher likelihoods of reproduction to genotypes whose phenotypes perform better on a specified task.

In 1993, Horner successfully demonstrated a system to automate timbre matching using a genetic algorithm \citep{horner1993machine}. The approach optimized the parameters of a fixed FM synthesis structure with a single modulator and multiple parallel carriers. Later, Johnson used an interactive genetic algorithm to control parameters for a granular synthesis algorithm \citep{johnson1999exploring} and Wehn used ideas from natural selection to explore the parameter space and in a more limited sense the connection space of FM-like synthesizers \citep{wehn1998using}. Genetic algorithms are well-suited for searching large, multi-dimensional spaces such as the parameters of audio synthesis algorithms and as such are a good choice for the task.

\paragraph{Genetic Programming}
John Koza's 1994 book defined a new class of genetic algorithms altogether \citep{koza1992genetic}. Koza borrowed concepts from traditional genetic algorithms to evolve computer programs and called the technique genetic programming (GP). The computer programs that GP evolves are represented as S-expressions as are programs in \audiolanguagenamelower. The set of functional and terminal expressions that GP can use to create programs are called primitives. GP can be adapted to many domains by the inclusion of domain-specific primitives. In 2000, Ricardo Garcia successfully demonstrated the technique of GP in the domain of target-matching audio synthesis algorithms \citep{garcia2000towards}.

Garcia's approach evolved trees that represented a set of instructions for building audio synthesis topologies such as the topology depicted in \ref{SINRENDERNETWORKFIG}. This indirect encoding was motivated by the fact that audio synthesis topologies can be cyclical and S-expressions cannot be. However, these non-cyclical ``instruction trees'' could contain instructions to build a cyclical synthesis topology and therefore allow unmodified genetic programming to be utilized to create cyclical synthesis algorithms.

Garcia made the following statement about his instruction trees ``every [instruction] tree maps into a single topology... but a topology can be represented by many different [instruction] trees'' \citep{garcia2012automatic}. This leads the author to believe that this approach is perhaps not the ideal way to apply GP to audio synthesis as this representation is searching through a larger space of S-expressions that map into a smaller space of synthesis algorithms. Hand-designed synthesis algorithms that utilize cyclic topologies to mimic acoustic instruments were rarely found in the research for this paper. Regardless, cyclic behavior could be simulated non-cyclically by the use of delay buffer primitives if it is found to be necessary.

\subsection{Experiment}
\subsubsection{Method Overview}\label{TMMETHOD}
This experiment will use genetic programming to evolve synthesis algorithms directly, utilizing a set of primitives identified as essential components for mimicking timbres. \audiolanguagenameupper{} programs contain many constants that contribute significantly to the overall sound of the algorithm. These will be optimized using the CMA-ES numerical optimization algorithm as GP is notoriously bad at solving problems that depend heavily on constant values.

The inputs to this system are a recorded target sound represented by a WAV file and the fundamental frequency of the target sound. This method differs from Garcia's whose representation did not involve the fundamental frequency as input. I argue that providing the fundamental frequency is a good addition to the inputs of the system as it prevents the algorithm from having to waste search time to find the fundamental and allows the experimenter to test how well the discovered algorithm generalizes to other pitches.

The experiment loop is as follows.

\begin{enumerate}
\item
Initialize population of \audiolaguagenamelower{} programs
\item
LOOP START:
Run each program and capture the rendered audio
\item
Evaluate the audio in comparison to the target by means of a fitness function
\item
If $best\ fitness\ found\ <\ threshold\ ||\ number of generations\ >=\ limit$, FINISH and return champion
\item
otherwise create the next generation of individuals using traditional genetic programming operations and return to LOOP START
\end{enumerate}

\subsubsection{Initialization}\label{TMINIT}
These are the primitives used in all timbre mimicking tests unless otherwise noted.
\begin{center}
\begin{tabular}{ | l | l | } \hline
Type & Arity \\ \hline
Addition & 2 \\ \hline
Multiplication & 2 \\ \hline
Frequency modulation oscillator & 2 \\ \hline
Time-varying filters & 3 \\ \hline
ADSR envelope & 1 \\ \hline
Sin wave oscillator & 0 \\ \hline
White noise & 0 \\ \hline
Ephemeral random constants & 0 \\ \hline
\end{tabular}
\end{center}
TODO: params
\subsubsection{Loop}

The random generation of the initial population is a mostly unmodified implementation of Koza's ramped half and half method used to generate diversity in the initial population \citep{koza1992genetic}. The only difference is that I do not check for duplicate S-expressions. The performance gain from not evaluating the same S-expression more than once does not seem to be worth the performance hit of comparing each new S-expression to all existing ones. I am using the technique of ephemeral random constants to propagate random values throughout the initial population.
TODO: params

\subsubsection{Evaluation}\label{TMEVAL}
The fitness function used for these tests was designed with the five parameters listed in \ref{GPINTRO} in mind. The target samples are partitioned into $m$ groups of $n$ samples with $o$ overlap between groups. These groups are analyzed using a discrete Fourier transform which can be used to determine magnitude and phase at $n/2$ frequency bins equally spaced between $0$ and $\frac{sample rate}{2}$ $Hz$.

The following values are calculated for each bin $j$ of group $i$:
\begin{itemize}
\addtolength{\itemindent}{1cm}
\vspace{-3mm}
\item $EMAC_{(i, j)}$: Exponential moving average value at the bin
\vspace{-3mm}
\item $TMAG_{(i, j)}$: Magnitude of the bin
\vspace{-3mm}
\item $TPHS_{(i, j)}$: Phase of the bin
\vspace{-3mm}
\item $UNDER_{(i, j)}$: Penalty for candidate undershooting the bin
\vspace{-3mm}
\item $OVER_{(i, j)}$: Penalty for candidate overshooting the bin
\vspace{-2mm}
\end{itemize}

The following values are calculated for each group $i$:
\begin{itemize}
\addtolength{\itemindent}{1cm}
\vspace{-3mm} 
\item $MAXABOVE_{(i)}$: $\max{(MAG_{(i, j)} - EMAC_{(i, j)})}$
\vspace{-3mm} 
\item $MAXBELOW_{(i)}$: $\max{(EMAC_{(i, j)} - MAG_{(i, j)})}$
\vspace{-2mm}
\end{itemize}

$UNDER_{(i, j)}$ and $OVER_{(i, j)}$ are calculated for each bin $j$ of group $i$ in the following way:
\vspace{-3mm} 
\begin{align*}
if (TMAG_{(i, j)} \le EMAC_{(i, j)}): \\
UNDER_{(i, j)} &= \frac{EMAC_{(i, j)} - TMAG_{(i, j)}}{MAXBELOW_{(i)}} * good + base \\
OVER_{(i, j)} &= \frac{EMAC_{(i, j)} - TMAG_{(i, j)}}{MAXBELOW_{(i)}} * bad + base \\
if (TMAG_{(i, j)} > EMAC_{(i, j)}): \\
UNDER_{(i, j)} &= \frac{TMAG_{(i, j)} - EMAC_{(i, j)}}{MAXABOVE_{(i)}} * bad + base \\
OVER_{(i, j)} &= \frac{TMAG_{(i, j)} - EMAC_{(i, j)}}{MAXABOVE_{(i)}} * good + base
\end{align*}
\vspace{-2mm}

Each candidate S-expression is rendered to produce a candidate wave form with the same number of samples as the target. The candidate samples are partitioned into groups of $n$ with $o$ overlap in the same way as the target samples.

The following values are calculated for each bin $j$ of group $i$ from the candidate samples:
\begin{itemize}
\addtolength{\itemindent}{1cm}
\vspace{-3mm} 
\item $CMAG_{(i, j)}$: Magnitude of the frequency bin
\vspace{-3mm} 
\item $CPHS_{(i, j)}$: Phase of the frequency bin
\vspace{-2mm}
\end{itemize}

Each candidate is assigned fitness in the following way:
\vspace{-2mm} 
\begin{align*}
PHSERROR_{(i, j)} &= (|TPHS_{(i, j)} - CPHS_{(i, j)}|)^{PHASEPENALTY} \\
if (CMAG_{(i, j)} \le TMAG_{(i, j)}): \\
MAGERROR_{(i, j)} &= (TMAG_{(i, j)} - CMAG_{(i, j)})^{UNDER_{(i, j)}} \\
if (CMAG_{(i, j)} > TMAG_{(i, j)}): \\
MAGERROR_{(i, j)} &= (CMAG_{(i, j)} - TMAG_{(i, j)})^{OVER_{(i, j)}}
\end{align*}
\vspace{-8mm}
\begin{align*}
FITNESS &= PHASEWEIGHT * (\sum_{i=1}^{m} \sum_{j=1}^{\frac{n}{2} - 1} PHSERROR_{(i, j)}) \\
        &+ MAGWEIGHT * (\sum_{i=1}^{m} \sum_{j=1}^{\frac{n}{2} - 1} MAGERROR_{(i, j)})
\end{align*}
\vspace{-4mm}

The motivations for this fitness function come from the attributes of timbre listed in \ref{GPINTRO}. At a high level, the weighting system is giving a higher penalty for a candidate overshooting a target valley and a higher penalty for a candidate undershooting a target peak. Penalty scaling is constant for a candidate undershooting a target valley and overshooting a target peak. This has the two-pronged effect of ensuring that better candidates in the early generations emphasize spectral peaks and don't try to match noisy bins. Later in the generations when the population begins to converge, the GA can continue to optimize by reducing the error at peaks and filling in lower magnitude spectral content.

This fitness function was empirically found to produce spectral content that roughly matched the sustain portion of targets but was worse at matching overall amplitude envelope. Experiments were performed where groups were weighted by their magnitude average's difference from the average magnitude across all groups. This placed higher fitness emphasis on transient groups, where the amplitude spectrum was the most different from the average (usually the attack/decay periods of the target). However, this technique was empirically found to lower the quality of the spectral matching.

TODO: params
\subsubsection{Population Maintenance}\label{TMMAINTENANCE}
The utilized genetic operations are those typically associated with genetic programming. Individuals are selected proportionally to their assigned fitnesses (where lower is better due to the fact that fitness is a calculation of error).
\begin{itemize}
\item
Crossover: Select two parents, swapping two randomly selected subtrees from within them and adding both resultant trees to the next generation. This operation is quite expressive as the heights of the resultant children can range from 0 to the sum of the heights of the parents. It can produce radically different children even if the parents are the same
\item
Reproduction: Select one algorithm and clone it without mutation into the next generation.
\item
Mutation: Select one algorithm and select one node $s$ from it uniformly at random. Replace this node with a new subtree with the $Grow$ method and max height of $maxheight - s.depth$.
\item
Numerical mutation: Select one algorithm and numerically mutate all of its coefficients.

TODO: params/include figures from Koza/elaborate on numerical mutation
\end{itemize}
\subsubsection{Experiments}\label{TMEXPERIMENTS}
\begin{comment}
Ideally one would ``clean'' the WAV file as much as possible by removing areas of silence, removing any DC offset and normalizing.
\end{comment}
\subsubsection{Results}\label{TMRESULTS}
\subsubsection{Conclusions}\label{TMCONCLUSIONS}
Because of the nature of S-expressions, a given node cannot have multiple parents. This theoretically does not limit the capabilities of the audio synthesis algorithms produced by GP because an S-expression can have multiple identical terminals that would be identical in behavior to a node with multiple parents. However, this greatly increases search complexity and computation time to match discover efficient algorithms such as FM synthesis using a single modulator and multiple carriers (the topology that \citep{horner1993machine} optimized using a genetic algorithm). Another limitation is lack of support for cyclic algorithms although as was hypothesized earlier this might not be of too much concern in the domain of target matching.
\subsubsection{Implementation Details and Parameters}\label{TMIMPLEMENTATION}
\begin{comment}
    // synth evolution parameters
    unsigned population_size;
    unsigned max_initial_height;
    unsigned max_height;
    bool erc;
    bool backup_all_networks;
    unsigned backup_precision;

    // fitness landscape parameters
    double best_possible_fitness;
    bool lower_fitness_is_better;

    // synth genetic parameters
    // numeric mutation
    double nm_proportion;
    double nm_temperature;
    unsigned nm_selection_type;
    double nm_selection_percentile;
    // mutation
    double mu_proportion;
    unsigned mu_type;
    unsigned mu_selection_type;
    double mu_selection_percentile;
    // crossover
    double x_proportion;
    unsigned x_type;
    unsigned x_selection_type;
    double x_selection_percentile;
    // reproduction
    double re_proportion;
    unsigned re_selection_type;
    double re_selection_percentile;
    bool re_reevaluate;
    // random new individual
    double new_proportion;
    unsigned new_type;

    // experiment parameters
    unsigned exp_ff_type;
    unsigned exp_suboptimize_ff_type;
    unsigned exp_suboptimize_type;
    unsigned exp_generations;
    double exp_threshold;

<?xml version="1.0" encoding="ISO-8859-1"?>
<Beagle>
  <Evolver>
    <BootStrapSet>
      <GA-InitCMAFltVecOp/>
      <AudioComparisonEvalOp/>
      <!-- <StatsCalcFitnessSimpleOp/> -->
      <TermMaxGenOp/>
      <!-- <MilestoneWriteOp/> -->
    </BootStrapSet>
    <MainLoopSet>
      <GA-MuWCommaLambdaCMAFltVecOp>
        <AudioComparisonEvalOp>
          <GA-MutationCMAFltVecOp>
            <SelectRandomOp/>
          </GA-MutationCMAFltVecOp>
        </AudioComparisonEvalOp>
      </GA-MuWCommaLambdaCMAFltVecOp>
      <!-- <StatsCalcFitnessSimpleOp/> -->
      <TermMaxGenOp/>
      <TermMinFitnessOp fitness="0.0"/>
      <GA-TermCMAOp/>
      <!-- <MilestoneWriteOp/> -->
    </MainLoopSet>
  </Evolver>
  <System>
    <Register>
      <Entry key="lg.console.level">0</Entry>
      <Entry key="lg.file.level">0</Entry>
      <Entry key="lg.file.name"></Entry>
      <Entry key="lg.show.class">0</Entry>
      <Entry key="lg.show.level">0</Entry>
      <Entry key="lg.show.type">0</Entry>
      <Entry key="ec.term.maxgen">2</Entry>
      <Entry key="ec.term.minfitness">0.0</Entry>
      <Entry key="ec.pop.size">10</Entry>
      <Entry key="ga.cmaes.mulambdaratio">2.0</Entry>
      <Entry key="ga.cmaes.mutpb">1.0</Entry>
    </Register>
  </System>
</Beagle>


\end{comment}


\section{Application 2: Interactive Evolution of Audio Synthesis Algorithms}
\subsection{Introduction}\label{IGAINTRO}
\subsection{Motivation}\label{IGAMOTIVATION}
\begin{quote}
\emph{To seek out new tonalities, new timbres... \\
To boldly listen to what no one has heard before.} \citep{sethares2004tuning}
\end{quote}
\begin{quote}
The musical value of the computer does not lie, of course, in its ability to duplicate exactly what a real instrument can do, but rather in yielding an extended repertory of sounds, including and going beyond the classes of sounds of actual instruments. \citep{risset1969analysis}.
\end{quote}

The concept of timbre is largely compartmentalized by modern composers. Acoustic composers typically restrict themselves to given sounds, such as the sounds that instruments in a wind ensemble or orchestra can produce. Electronic and electro-acoustic composers have a wider timbral palette to work with but usually restrict themselves to working within the confines of available synthesizers and samples. It has a hard task to discover novel sounds from thoroughly investigated synthesis techniques. Wishart said in his book Audible Design \citep{wishart1994audible}:
\begin{quote}
The spectral characteristics of sound have, for so long, been inaccessible to the composer that we have become accustomed to lumping together all aspect of the spectral structure under the catch all term ``timbre''...
\end{quote}	 
These issues suggest the need for alternative and accessible methods of timbre investigation for musicians and researchers. Such capabilities could theoretically allow a musician to utilize an instrument they do not have physical access too from a single recording of its sound. Many electronic musicians utilize synthesis techniques such as additive, subtractive, ring modulation, amplitude modulation, and frequency modulation to create sounds. This paper will also present an interactive genetic algorithm for searching the space of audio synthesis algorithms of which these traditional synthesis techniques are a minuscule subset of the search space.

\subsection{Experiment}
\subsubsection{Method}\label{IGAMETHOD}
This experiment will use genetic programming to evolve synthesis algorithms directly, utilizing any set of primitives that the user chooses. The inputs to this system will be the user-selected primitives and the output will be multiple ASA's for the user to modify and evaluate. The user will be able to step through the evolutionary process, assigning fitness and auditioning the produced ASA's using a MIDI keyboard or virtual MIDI keyboard. The user may manipulate any variable constants of the produced ASA. The user may opt to input an audio file and let the synthesizer attempt to evolve towards it. The user may also let evolution experiment randomly using novelty search with unencountered fundamental/partial frequency ratios as the novelty goal.
\subsubsection{Initialization}\label{IGAINIT}
\subsubsection{Evaluation}\label{IGAINIT}
\subsubsection{Population Maintenance}\label{IGAMAINTENANCE}
\subsubsection{Conclusions}\label{IGACONCLUSIONS}
\subsubsection{Implementation Details and Parameters}\label{IGAIMPLEMENTATION}

An early prototype for the interactive system was built using Cabbage, Csound and Python. After several performance limitations with this setup were encountered the implementation was switched to C++ and utilized the JUCE audio/GUI library. JUCE is a popular open-source library for developing audio plug-ins and was used as the core audio engine of the implementation. This project could not have been possible without JUCE and author/maintainer Jules's relentless demand for quality. Filter primitives were developed by modifying Tony Fisher's filter design C program. Spectrum analysis for the fitness function was performed by Kiss FFT, a relatively fast and extremely lightweight implementation of the Fast Fourier transform.

Target-matching tests were run on the UT Condor Computing Cluster. \begin{comment}Interactive evolution tests were run on a variety of different machines. TODO: fill in with actual data \end{comment}

\section{Future Work}
The system will also allow the user to examine, test and adjust these discovered algorithms to improve their fit.

\subsection{Language Enhancements}\label{FUTUREENHANCEMENTS}
\subsection{Additional Primitives}\label{FUTUREPRIMITIVES}
\subsection{New Applications}\label{FUTUREAPPLICATIONS}

\section{Acknowledgements}
The author would like to thank the Computational Intelligence in Game Design lab, part of the University of Texas's Freshman Research Initiative program, for providing the tools and workspace to implement this project. Thanks to Joel Lehman for advice on the genetic algorithm aspects of this project. Thanks to Ashvin Bashyam for his advice on the subtleties of working in a frequency domain representation of a signal. Thanks to Jules, creator of the JUCE C++ toolkit, for his attention to quality making this project a pleasure to implement. Finally, the author wishes to express gratitude to his advisors and committee members who were extremely helpful throughout the process.

\section{Links}
\begin{itemize}
\item{\url{http://www.synthax.tk}}
\item{\url{http://www.juce.com/}}
\item{\url{http://www.steinberg.net/en/products/vst.html}}
\end{itemize}

\bibliographystyle{abbrv}
\bibliography{bibthesis}

\end{document}
